%% This work is licensed under the Creative Commons Attribution-NonCommercial-NoDerivs 3.0 Unported License.
%% To view a copy of this license, visit www.creative-commons.org/licenses/by-nc-nd/3.0 or send a letter
%% to Creative Commons, 444 Cas-tro Street, Suite 900, Mountain View, California, 94041, USA.

%% The above statement applies to all included files too!!!

\documentclass[a4paper,12pt]{article}

\usepackage[
	%iwona,
	%final,
	blacklogo,
	%joinlists,
	abbreviations,
	figures,
	tables,
	listings,
	language=czech,
	bibfile=bibliography.bib,
	bibencoding=utf8,
	bibstyle=numeric,
	encoding=utf8,
	]{upbase}

\usepackage[
	custompicture=graphics/elephant.png
	]{upsimple}

\uptitle{Databázové systémy 1}
\upsubtitle{KMI/DATA1}
\upyear{\today}
\upauthor{Martin Rotter}
\upanot{Tento dokument obsahuje přepisy přednášek, které vedl \href{mailto:vilem.vychodil@upol.cz}{doc. RNDr Vilém Vychodil, Ph.D.}. Uvedená práce (dílo) podléhá licenci \href{http://creativecommons.org/licenses/by-nc-nd/3.0/}{Attribution-NonCommercial-NoDerivs 3.0 Unported}, a to včetně vložených souborů, to neplatí pro loga jiných společností, jako je například logo \href{http://www.postgresql.org/}{PostgreSQL}.}
\uppapertype{P\v{R}EPISY P\v{R}EDN\'{A}\v{S}EK}

\begin{document}

% vytiskne titulní stranu
\upmaketitle

% vytiskne anotaci (abstrakt)
\upmakeanot

% tiskne obsah a seznamy obrázku, tabulek a případně zdr. kódů, více v makrech \uptableofcontents a \upprintlists
% tiskne pouze povolené seznamy
\uptocandlists

\section{Úvod}
Toto je dokumentace stylu \keyword{upsimple}.

\upabbrevdeclare{UPOL}{UPOL}{Univerzita Palackého v Olomouci}

Pokud například máme definovanou zkratku UPOL, tak se na ní můžeme odkázat několikrát, třeba dvakrát, za sebou. Tedy \upabbrevref{UPOL} a ještě jednou \upabbrevref{UPOL},

\index{hm}.

% vytiskne nepatrnou mezeru do obsahu dokumentu (používá se pro vizuální oddělení) dodatečných částí práce
% od hlavní části práce)
\upendoftreatise

% tiskne přílohy
\upappendix
\section{První příloha}
Text první přílohy

\section{Druhá příloha}
Text druhé přílohy

\upendoftreatise

% pokud některou z následujích featur nepotřebujete tak ji jednoduše zakomentujte a případně zakažte danou funkcionalitu
% viz parametr abbreviatons

% vytiskne seznam zkratek
%\upprintabbrevlist

% tiskne seznam teorémů
\upprinttheoremlist

% vytiskne bibliografii
\upprintbibliography

% vytiskne rejstřík
%\upprintindex

\end{document}