\subsubsection{Relační dělení}
Dělení je protipólem projekce a značí se \enquote{$\div$}. Bývá odpovědí na dotazu typu: \enquote{Najdi osobu, která obsolvovala každý kurz.} Určitá n-tice $r$ se nachází v tabulce $r \in \pi_{R} (\mathcal{D})$ právě, když existuje $s$ tak, že $rs \in \mathcal{D}$. Dělení lze interpretovat jako odpověď na univerzální dotazy.

Uvažujme relace:
\begin{enumerate}
\item $\mathcal{D}_{1}$ na schématu $R$, které říkejme \textit{dělenec}.
\item $\mathcal{D}_{2}$ na schématu $S$, které říkejme \textit{dělitel}.
\item $\mathcal{D}_{3}$ na schématu $T$, které říkejme \textit{prostředník}.
\end{enumerate}
Výsledkem dělení $\mathcal{D}_{1} \div_{\mathcal{D}_{3}} \mathcal{D}_{2}$, tedy tzv. podílem je relace nad schématem $R \cap T$. Matematicky:
\begin{align*}
\mathcal{D}_{1} \div_{\mathcal{D}_{3}} \mathcal{D}_{2} &= \left\{ r(R \cap T) \; | \; r \in \mathcal{D}_{1} \text{ tak, že } \forall s \in \mathcal{D}_{2} \text{ takové, že } r(R \cap S \cap T) \right. \\
&\left.= s(R \cap S \cap T) \text{ platí, že } (rs) ((R \cup S) \cap T) \in \pi_{(R \cup S) \cap T} (\mathcal{D}_{3}) \right\}
\end{align*}
Pokud $R \cap S = \varnothing$, pak platí, že:
\begin{align*}
\mathcal{D}_{1} \div_{\mathcal{D}_{3}} \mathcal{D}_{2} = &\left\{ r(R \cap T \; | \;  r \in \mathcal{D}_{1} \text{ tak, že pro } \forall s \in \mathcal{D}_{2}  \text{ platí, že } \right. \\
&\left.(rs) \left( \left(R \cup S\right) \cap T\right) \in \pi_{(R \cup S) \cap T} (\mathcal{D}_{3}) \right\}
\end{align*}
Dělení trošku jinak: Pro $ T = R \cup S$ 
$$
\mathcal{D}_{1} \div_{\mathcal{D}_{3}} \mathcal{D}_{2} = \left\{ r \; | \; r \in \mathcal{D}_{1} \text{ tak, že pro } \forall s \in \mathcal{D}_{2} \text{ platí, že } rs \in \mathcal{D}_{3} \right\}
$$
Nebo také: (Vhodné pro vyjádření v SQL)
$$
\mathcal{D}_{1} \div_{\mathcal{D}_{3}} \mathcal{D}_{2} = \pi_{R \cap T} \left( \mathcal{D}_{1} \right) \setminus \pi_{R \cap T} \left( \left( \pi_{R \cap T} \left( \mathcal{D}_{1} \right) \Join \pi_{S \cap T} \left( \mathcal{D}_{2} \right) \right) \setminus \pi_{(R \cup S) \cap T} \left( \mathcal{D}_{3}  \right) \right) %v sesite mam index 2%
$$
\begin{upcode}{Relační dělení (Tutorial D)}{}{TutorialD}
relexpr1 DIVIDE BY relexpr2 PER relexpr3;
\end{upcode}

\subsubsection{Vnitřní a vnější spojení}
Doposud probrané spojení bylo vnitřní spojení (klíčové slovo\upinlinecode{SQL}{!}{INNER JOIN} v SQL). Abychom mohli uvažovat vnější spojení, je třeba uvažovat koncept \textit{chybějící} hodnoty, to jest porušit první normální formu. V SQL by se tedy místo\upinlinecode{SQL}{!}{NATURAL (INNER) JOIN} použilo\upinlinecode{SQL}{!}{FULL OUTER JOIN},\upinlinecode{SQL}{!}{LEFT OUTER JOIN} nebo\upinlinecode{SQL}{!}{RIGHT OUTER JOIN}.
\begin{upcode}{Vnější spojení (SQL)}{}{SQL}
SELECT * FROM table_1 FULL OUTER JOIN ON table_1.x < table_2.x
\end{upcode}
Při levém vnějším spojení se do výsledku přidávají i visící n-tice z první tabulky.

\section{Neznámé hodnoty}
Drtivá většina \enquote{moderních} \upabbrevref{DBMS} podporuje tzv. tříhodnotovou logiku, kde vstupuje do popředí, kromě \textit{pravdy} a \textit{nepravdy}, také hodnota \textit{NULL}, která říká, že daná hodnota je nedefinovaná nebo neznámá, což se bude jevit každému myslícímu člověku jako absolutní nesmysl. Těmto hodnotám je dobré se vyhnout, protože činí logické operace (potažmo funkce) nepřehlednými, jak ukazuje tabulka \ref{tab:operace}.

\begin{table}
\caption{Logické funkce}\label{tab:operace}
\begin{subtable}[t]{0.45\textwidth}
\centering
\caption{Logický součin}
\begin{tabular}{c || c c c}
$\land$ & 0 & N & 1 \\
\hline
0 & 0 & 0 & 0 \\
N & 0 & N & N \\
1 & 0 & N & 1
\end{tabular}
\end{subtable}
~
\begin{subtable}[t]{0.45\textwidth}
\centering
\caption{Logická implikace}
\begin{tabular}{c || c c c}
$\to$ & 0 & N & 1 \\
\hline
0 & 1 & 1 & 1 \\
N & N & N & 1 \\
1 & 0 & N & 1
\end{tabular}
\end{subtable}
\end{table}
\begin{upexample}[Datův příklad k NULLovým hodnotám]
Mějme vstupní data (tabulka \ref{tab:null}) a nad těmito daty proveďme dotaz:
\begin{table}
\caption{Příklad relace s NULL hodnotou}\label{tab:null}
\begin{subtable}[t]{0.45\textwidth}
\centering
\caption{Tabulka n1}
\begin{tabular}{c | c}
x & yn \\
\hline
45 & London
\end{tabular}
\end{subtable}
~
\begin{subtable}[t]{0.45\textwidth}
\centering
\caption{Tabulka n2}
\begin{tabular}{c | c}
y & zn \\
\hline
33 & -
\end{tabular}
\end{subtable}
\end{table}
\begin{upcode}{Dotaz pracující s NULL hodnotami (SQL)}{}{SQL}
SELECT	x, y
FROM	n1, n2
WHERE	yn <> zn OR zn <> "Paris";
\end{upcode}
Logicky bychom očekávali jako výsledek operace n-tici $\langle 45,33 \rangle$, avšak v SQL tomu tak není, protože dané porovnání v dotazu nevrací nepravdu nebo pravdu, ale hodnotu NULL.
\end{upexample}

\begin{quote}
\foreignlanguage{english}{\enquote{You can never trust the answer you get from database with nulls.}}
\end{quote}
\hfill C.~J.~Date

\section{Integritní omezení}
Jedná se o jednu z komponent Coddova relačního modelu. Účel integritních omezení je ten, že je v databázi třeba udržet určitou efektivitu operací a hlavně uchovat data v konzistentním stavu z hlediska jejich smyslu.

\subsection{Výrazy relační algebry}
\upabbrevref{RA} obsahuje atomické výrazy, tedy relační proměnné a konstantní relace (TABLE\_DEE, TABLE\_DUM). Dále jsou obsaženy také složené výrazy, které jsou ustaveny rekurzivně, tedy:
\begin{enumerate}
\item Pokud jsou $R_{1}$ a $R_{2}$ relační výrazy.
\item Pak je i $R_{1} \Join R_{2}$ relační výraz.
\item A taktéž i $R_{1} \cap R_{2}$ je relační výraz.
\item Podobně s dalšími operacemi.
\end{enumerate}

\subsection{Omezení z pohledu relační algebry}
\begin{enumerate}
\item\label{omez:1} Pro výraz $R$ definujeme omezení tvaru $R = \varnothing$, tedy nesmí existovat n-tice, které by jej splňovala.
\item\label{omez:2} Pro výrazy $R, S$ definujeme omezení tvaru $R \subseteq S$, tedy každá n-tice splňující $R$ musí splňovat $S$.
\end{enumerate}
Pro \ref{omez:1} a \ref{omez:2} existuje vzájemná převoditelnost:
\begin{align*}
\text{\enquote{$\Rightarrow$}}&, R = \varnothing \text{ lze chápat jako } R \subseteq \varnothing \\
\text{\enquote{$\Leftarrow$}}&, R \subseteq S \text{ lze chápat jako } R \setminus S = \varnothing
\end{align*}

\subsection{Referenční integritní omezení}
Výchozí myšlenkou je, že pokud máme systém tabulek, tak hodnoty v určitých sloupcích, které popisují vlastnost určité entity, se musí shodovat s hodnotami ve sloupcích (popisujících stejnou vlastnost) pro stejnou entitu v ostatních tabulkách v tomto systému tabulek.

Tuto vlastnost nazýváme \textit{cizí klíč}.

\begin{upexample}[Cizí klíče v relacích]\label{ex:cizi}
Představme si dvojici tabulek (systém tabulek \ref{tab:cizi}), která popisuje vlastnictví aut jednotlivými akademickými pracovníky.
\begin{table}
\caption{Cizí klíče v relacích}\label{tab:cizi}
\begin{subtable}[t]{0.6\textwidth}
\centering
\caption{Pracovnící}\label{tab:vl1}
\begin{tabular}{c || c c c}
id & jméno & příjmení & laboratoř \\
\hline
\cellcolor{red}01 & Vilík & Devil & 666 \\
\cellcolor{yellow}02 & James & Bond & 007 \\
03 & Kryštof & Kolumbus & 1492 \\
\cellcolor{green}04 & Jiří & Dikobraz & 1111
\end{tabular}
\end{subtable}
~
\begin{subtable}[t]{0.38\textwidth}
\centering
\caption{Vlastníci aut}\label{tab:vl2}
\begin{tabular}{c || c}
pracovnik\_id & auto \\
\hline
\cellcolor{green}04 & Trabant \\
\cellcolor{yellow}02 & Porsche 911 \\
\cellcolor{red}01 & Nimbus 3000 \\
\cellcolor{green}04 & Autobus
\end{tabular}
\end{subtable}
\end{table}
Dle tabulky vidíme, že číslo každého vlastníka auta v tabulce \ref{tab:vl2} odpovídá osobě z tabulky \ref{tab:vl1}.

Zároveň neexistuje takový majitel auta, který by neměl v první tabulce záznam. Auto zkrátka nemůže být v tomto idealizovaném systému bez majitele. Tento vztah lze zapsat matematicky. Pojmenujme první tabulku $\mathcal{D}_{1}$ a druhou $\mathcal{D}_{2}$, pak lze napsat následující tvrzení:
$$
\rho_{id \leftarrow pracovnik\_id}(\pi_{\left\{ pracovnik\_id \right\}} \left( \mathcal{D}_{2} \right) ) \subseteq \pi_{\left\{ id \right\}} \left( \mathcal{D}_{1} \right)
$$
\end{upexample}

V SQL lze modelovat cizí klíče pomocí klíčového slova\upinlinecode{SQL}{!}{FOREIGN KEY}.

\subsubsection{Chování při použití cizího klíče}
Konkrétní tvrzení z příkladu \ref{ex:cizi} lze pochopitelně zobecnit:
$$
\rho_{y \leftarrow x}(\pi_{\{x\}} \left( R_{1} \right) ) \subseteq \pi_{\{y\}} \left( R_{2} \right)
$$
\begin{enumerate}
\item Nastane chyba, pokud se pokusíme do $R_{1}$ vložit n-tici, jejíž hodnota pro atribut $x$  se nenachází v žádné n-tici z $R_{2}$ v atributu $y$.
\item Pokus o aktualizaci cizího klíče, kde mohou nastav v zásadě dvě možné situace:
\begin{enumerate}
\item Pokus smazat záznam z $R_{2}$, jehož hodnota pro $y$ se nachází v $R_{1}$. Tedy v analogii k předchozímu příkladu \ref{ex:cizi} bychom se snažili smazat pracovníka, který ale vlastní nějaké auto.
\item Pokus změnit záznam z $R_{2}$ tak, že nová hodnota pro $y$ se nenachází v $R_{1}$. Tedy v analogii k předchozímu příkladu \ref{ex:cizi} bychom se snažili změnit vlastníka nějakého auta na pracovníka, který neexistuje.
\end{enumerate}
\end{enumerate}

\upabbrevref{DBMS} může určitým způsobem reagovat na předchozí situace:
\begin{enumerate}
\item Databázový systém nemusí vykonat nic, jen nahlásí chybu. Tohle chování je vlastní systémům pracujícím s SQL.
\item Může být použito \textit{kaskádování}, kde se provedená změna propaguje do ostatních tabulek, ve kterých je změněná hodnota referována jako cizí klíč.
\item Databázový systém může rovněž nastavit výskyt cizího klíče buď na NULL nebo na výchozí hodnotu (viz. klíčove slovo\upinlinecode{SQL}{!}{DEFAULT} v SQL).
\end{enumerate}

% TODO: omezení pomocí klíčů

\section{Tranzitivní uzávěr relace}
\begin{uptheorem}[Tranzitivní uzávěr relace (v matematickém smyslu)]
Mějme $R \subseteq A \times A$. Pak tranzitivní uzávěr $R^{\infty} \subseteq A \times A$ je definován jako:
$$
R^{\infty} = \bigcup_{n = 1}^{\infty} R^{n} = \bigcup_{n = 1}^{\infty} \underbrace{R \circ \cdots \circ R}_{\text{n-krát}}
$$
Pokud je $A$ je konečná, pak existuje index $m$ tak, že $R^{\infty} = \bigcup_{n = 1}^{m} R^{n}$. Je-li  navíc $R$ reflexivní, pak $R^{\infty} = R^{m}$.
\end{uptheorem}
Nyní aplikujme tranzitivní uzávěr na databáze.
\begin{upexample}[Tranzitivní uzávěr tabulky]
Mějme tabulku (tabulka \ref{tab:predmety1}) podmi\-/ňujících a podmíněných předmětů se dvěma sloupci. Výsledkem tranzitivního uzá\-věru je úplná informace o podmiňujích atributech. Tabulky zobrazují podmi\-/ňující a podmíněné předměty. Například vidíme, že pro splnění předmětu PP3 je třeba mít přímo splněný předmět PP2, avšak ke splnění předmětu PP2 je třeba mít splněn VT i PP1. Tyto informace jdou z tabulky číst relativně obtížně a pro gigantické tabulky se nepřímé čtení těchto vztahů stává problémové.
\begin{table}
\caption{Tranzitivní uzávěr tabulky}\label{tab:predmety}
\begin{subtable}[t]{0.45\textwidth}
\centering
\caption{Výchozí tabulka}\label{tab:predmety1}
\begin{tabular}{c | c}
podmiňující př. & podmíněný př. \\
\hline
\rowcolor{green}PP1 & PP2 \\
\rowcolor{red}VT & PP2 \\
\rowcolor{yellow}PP2 & PP3
\end{tabular}
\end{subtable}
~
\begin{subtable}[t]{0.45\textwidth}
\centering
\caption{Tranizitivně uzavřená tabulka}\label{tab:predmety2}
\begin{tabular}{c | c}
podmiňující př. & podmíněný př. \\
\hline
\rowcolor{green}PP1 & PP2 \\
\cellcolor{green}PP1 & \cellcolor{yellow}PP3 \\
\rowcolor{yellow}PP2 & PP3 \\
\rowcolor{red}VT & PP2 \\
\cellcolor{red}VT & \cellcolor{yellow}PP3
\end{tabular}
\end{subtable}
\end{table}
\end{upexample}
Pokud je $\mathcal{D}$ relace nad schématem $\lbrace x, y \rbrace$, která je protějškem $R \subseteq A \times A$, pak
$$
\mathcal{D} = \left\{ \left\{ \langle x, a \rangle, \langle x, b \rangle \right\} \; | \; \langle a, b \rangle \in R \right\}
$$
a vyjídříme protějšek $R^{\infty}$ jako
$$
\mathcal{D}^{\infty} = \bigcup_{n = 1}^{m} \pi_{\left\{ x, y \right\}} \left( \Join_{i = 1}^{n} \rho _{x'_{i, n} \leftarrow x, y'_{i, n} \leftarrow y}(\mathcal{D}) \right)
$$
kde
$$
\begin{array}{l l}
x'_{1, n} = x & \text{ pro každé } n \\
y'_{m, n} = y & \text{ pro každé } n \\
x'_{i, n} = y'_{i-1,n} & \text{ pro každé } 2 \leq i \leq n
\end{array}
$$
Tranzitivní uzávěr je definovatelný, ale definice(vyjádření) závisí na $m$.
\begin{upcode}{Tranzitivní uzávěr (Tutorial D)}{}{TutorialD}
TCLOSE relexpr
\end{upcode}
\begin{upcode}{Kostra rekurzivního dotazu (SQL)}{code:recursive}{SQL}
WITH RECURSIVE jmeno(sloupce) AS (
	/* nerekurzivní výraz
	SELECT	*
	FROM	foo
	UNION	DISTINCT
	/* rekurzivní výraz, který může používat jmeno */
	SELECT	*
	FROM 	bar;
)
SELECT	*
FROM	jmeno_sloupce;
\end{upcode}
\begin{upcode}{Tranzitivní uzávěr (SQL)}{}{SQL}
WITH RECURSIVE tr(x, y) AS (
	SELECT	*
	FROM 	r
	UNION	DISTINCT
	SELECT	x, y tr.y FROM r, tr
	WHERE	r.y = tr.x
)
SELECT * FROM tr;
\end{upcode}
Vyhodnocení dotazu z kódu \ref{code:recursive} probíhá následovně:
\begin{enumerate}
\item Vyhodnotí se nerekurzivní dotaz a jeho výsledky se uloží do dočasných tabulek RESULT a WORK.
\item Dokud je WORK neprázdná, tak se opakuje následující:
\begin{enumerate}
\item Vyhodnotí se rekurzivní dotaz, v němž je self reference nahrazena tabulkou WORK. Výsledek si označme $R_{1}$.
\item Pokud je použit\upinlinecode{SQL}{!}{UNION DISTINCT}, pak se z $R_{1}$ ostraní duplicity a vše, co už je v RESULT.
\item Obsah $R_{1}$ se přidá do RESULT.
\item WORK se nastaví na $R_{1}$.
\end{enumerate}
\end{enumerate}