\subsubsection{Sumarizace (agregace)}
\begin{upcode}{Počet n-tic v relaci (SQL)}{}{SQL}
SELECT COUNT(*) FROM my_table;
\end{upcode}
Výsledkem obdobného dotazu v SQL bude vždy relace. Například v tomto přípa-dě relace o jedné jednoprvkové n-tici, která obsahuje počet řádků původní relace.
\begin{upcode}{Počet n-tic v relaci (Tutorial D)}{}{TutorialD}
COUNT(relexpr)
\end{upcode}
Naopak v Tutorial D nebude výsledkem relace, avšak skalární hodnota. Tedy jednoduše číslo. Chování jazyka SQL lze v Tutorial D umitovat.
\begin{upcode}{Imitace chování SQL (Tutorial D)}{}{TutorialD}
SUMMARIZE relexpr ADD (COUNT() AS count_of_tuples)
\end{upcode}
\begin{upcode}{Pokročilá sumarizace (Tutorial D)}{code:summarized}{TutorialD}
SUMMARIZE relexpr1
PER (relexpr2)
ADD (summary AS name)

/* místo PER lze použít také BY */
/* platí, že BY {seznam atributů} odpovídá PER (relexpr1 {seznam atributů}) */
SUMMARIZE relexpr1
BY (seznam atributů)
ADD (summary AS name)
\end{upcode}
Navíc by mělo platit, že\upinlinecode{TutorialD}{!}{relexpr1} $\subseteq$\upinlinecode{TutorialD}{!}{relexpr2}. Význam zdrojového kódu \ref{code:summarized} je takový, že jdeme přes všechny n-tice z\upinlinecode{TutorialD}{!}{relexpr2} a počítáme sumarizaci ze všech n-tic z\upinlinecode{TutorialD}{!}{relexpr1}, které se shodují na atributech z relačního schématu z\upinlinecode{TutorialD}{!}{relexpr2}.

\subsubsection{Seskupování}
Uplatňuje se v modelech, které umožňují atributy s doménami, které jsou množi-ny relací.

\subsubsubsection{Metoda UNGROUP}
\begin{upcode}{Seskupování UNGROUP (Tutorial D)}{}{TutorialD}
/* atributy jsou typu relace */
relexpr UNGROUP (atribut, dalsi_atribut)
\end{upcode}
Výsledkem je relace nad schématem, které vznikne tak, že místo atributů ze seznamu budeme mít atributty vnořených tabulek a výsledek obsahuje n-tice tak, že pro každou n-tici je ve výsledku (obecně) několik n-tic, jejichž hodnoty jsou brány z tabulek ve výchozí n-tici metodou každý s každým.

\subsubsubsection{Metoda GROUP}
\begin{upcode}{Seskupování GROUP (Tutorial D)}{}{TutorialD}
/* atributy jsou typu relace */
relexpr GROUP (atribut, dalsi_atribut)
\end{upcode}
Na základě relace a seznamu atributů množin atributů, které se mají seskupit pod danými novými jmény se vytvoří relace, ve které jsou zbylé atributy výchozí relace a nové atributy, které nabývají hodnot, v nichž jsou maximální relace obsahující n.tice hodnot výchozí tabulky tak, že individuální hodnoty výsledných n-tic jsou v relaci se vzniklými relacemi právě tehdy, když zřetězení libovolných individuálních n-tic se zbylými individuálními hodnotami z každé výsledné n-tice se nachází ve výchozí tabulce.

\subsubsection{Přejmenování}
Z relace $\mathcal{D}$ nad relačním schématem $R$ se vytvoří nová relace
$$
\rho \; y_{1}' \Leftarrow y_{1}, \ldots, y_{n}' \Leftarrow y_{n} (\mathcal{D})
$$
nad schématem $R$, ve kterém byly atributy $y_{1}, \ldots, y_{n}$ přejmenovány na atributy $y_{1}', \ldots, y_{n}'$, ale za předpokladu, že každé dva atributy $y_{i}, y_{i}' \text{ pro } 1 \leq i \leq n$ mají stejný typ.
\begin{upcode}{Přejmenování (Tutorial D)}{}{TutorialD}
relexpr RENAME (old AS new, old_2 AS new_2))
\end{upcode}
\begin{upcode}{Přejmenování (SQL)}{}{SQL}
SELECT old AS new, old_2 AS new_2 FROM table_1;
\end{upcode}

\subsubsection{Přirozené spojení}
\begin{uptheorem}[Přirozené spojení]
Mějme relace $\mathcal{D}_{1}$ nad schématem $R \cup S$ a $\mathcal{D}_{2}$ nad schématem $S \cup T$ tak, že $R, S, T$ jsou po dvou disjunktní a $S$ je množina všech společných atributů. Pak je přírozené spojení $\mathcal{D}_{1}, \mathcal{D}_{2}$ relace nad schématem $R \cup S \cup T$ dáno předpisem
$$
\mathcal{D}_{1} \Join \mathcal{D}_{2} = \left\{ rst \; | \; rs \in \mathcal{D}_{1}, st \in \mathcal{D}_{2} \right\}
$$
\end{uptheorem}
\begin{upcode}{Sjednocení n-tic (Tutorial D)}{}{TutorialD}
/* výsledkem je TUPLE {x 10, y 20, z 30} */
TUPLE {x 10, y 20} UNION TUPLE {y 20, z 30}
\end{upcode}
\begin{upexample}[Přirozené spojení]
Přírozeným spojením získáme tabulku, který má společné záhlaví, avšak obsahuje pouze n-tice, které se shodují na společných atributech daných vstupních tabulek resp. relací. Názorně na tabulkách ve skupině tabulek \ref{tab:prir_spojeni}.
\begin{table}
\caption{Přirozené spojení tabulek}\label{tab:prir_spojeni}
\begin{subtable}[t]{0.3\textwidth}
\centering
\caption{První operand}
\begin{tabular}{c | c | c}
w & x & y \\
\hline
1 & 1 & 2 \\
2 & \cellcolor{red}2 & \cellcolor{red}2 \\
3 & \cellcolor{yellow}3 & \cellcolor{yellow}2 \\
4 & 3 & 4 \\
5 & \cellcolor{green}5 & \cellcolor{green}5
\end{tabular}
\end{subtable}
~
\begin{subtable}[t]{0.3\textwidth}
\centering
\caption{Druhý operand}
\begin{tabular}{c | c | c}
x & y & z \\
\hline
1 & 1 & 2 \\
\cellcolor{red}2 & \cellcolor{red}2 & 1 \\
\cellcolor{yellow}3 & \cellcolor{yellow}2 & 4 \\
4 & 4 & 5 \\
\cellcolor{green}5 & \cellcolor{green}5 & 5
\end{tabular}
\end{subtable}
~
\begin{subtable}[t]{0.3\textwidth}
\centering
\caption{Výsledná tabulka}
\begin{tabular}{c | c | c | c}
w & x & y & z \\
\hline
2 & \cellcolor{red}2 & \cellcolor{red}2 & 1 \\
3 & \cellcolor{yellow}3 & \cellcolor{yellow}2 & 4 \\
5 & \cellcolor{green}5 & \cellcolor{green}5 & 5
\end{tabular}
\end{subtable}
\end{table}
\end{upexample}
Libovolné n-tice $r$ nad schémater $R$ a $s$ nad schématem $S$ nazveme \textit{spojitelné}, pokud projekce $r (R \cup S) = s (R \cup S)$. Tedy pokud mají stejné hodnoty ve společných atributech. Přirozené spojení odpovídá množině všech spojitelných n-tic z nějakých vstupních relací $\mathcal{D}_{1}, \mathcal{D}_{2}$.
\begin{upcode}{Přirozené spojení (Tutorial D)}{}{TutorialD}
relexpr1 JOIN relexpr2
\end{upcode}
\begin{upcode}{Přirozené spojení (SQL)}{}{SQL}
SELECT * from table_1 NATURAL JOIN table_2;

/* případně */
SELECT	table_1.*, table_2.*
FROM	table_1, table_2
WHERE	table_1.x = table_2.x AND table_1.y = table_2.y;
\end{upcode}
\begin{uptheorem}[Visící n-tice]
Pokud uvažujeme spojení $\mathcal{D}_{1} \Join \mathcal{D}_{1}$, pak se $rs \in \mathcal{D}_{1}$ nazývá \textit{visící} n-tice, pokud $rs \notin \pi_{R \cup S} (\mathcal{D}_{1} \Join \mathcal{D}_{2})$, tedy právě, když $rs$ není spojitelné se žádnou n-ticí z $\mathcal{D}_{2}$.
\end{uptheorem}
\begin{uptheorem}[Úplné přirozené spojení]
Pokud relace $\mathcal{D}_{1}, \mathcal{D}_{2}$ nemají vzhledem ke svému přirozenému spojení žádné visící n-tice, pak lze toto přirozené spojení označit jako \textit{úplné}.
\end{uptheorem}

\subsubsubsection{Speciální případy přirozeného spojení}
Předpokládejme, že $\mathcal{D}_{1}$ je relace nad schématem $R \cup S$ a $\mathcal{D}_{2}$ je relace nad schématem $S \cup T$. Schémata $R, S, T$ jsou pod dvou disjunktní. 
\begin{description}
\item[Průnik] Jedná se o speciální případ pro $R = \varnothing \text{ a } T = \varnothing$, pak obě relace $\mathcal{D}_{1} \text{ a } \mathcal{D}_{2}$ jsou situovány nad schématem $S$. Následně spojení:
\begin{align*}
\mathcal{D}_{1} \Join \mathcal{D}_{2} &= \left\{ rst \; | \; rs \in \mathcal{D}_{1} \text{ a } st \in \mathcal{D}_{2} \right\} \\
&= \left\{ \varnothing s \varnothing \; | \; \varnothing \in \mathcal{D}_{1} \text{ a } s \varnothing \in \mathcal{D}_{2} \right\} \\
&= \left\{ s \; | \; s \in \mathcal{D}_{1} \text{ a } s \in \mathcal{D}_{2} \right\} \\
&= \mathcal{D}_{1} \cap \mathcal{D}_{2}
\end{align*}

\item[Kartézský průnik] Jedná se o spojení, kde společné schéma (seznam atributů) $S = \varnothing$. Následně:
\begin{align*}
\mathcal{D}_{1} \Join \mathcal{D}_{2} &= \left\{ rst \; | \; rs \in \mathcal{D}_{1} \text{ a } st \in \mathcal{D}_{2} \right\} \\
&= \left\{ rt \; | \; r \in \mathcal{D}_{1} \text{ a } t \in \mathcal{D}_{2} \right\}
\end{align*}
Tento případ spojení značíme $\boxtimes$. Velikost výsledku tohoto spojení, tedy $\left| \mathcal{D}_{1} \boxtimes \mathcal{D}_{2} \right| = \left|\mathcal{D}_{1}\right| \cdot \left|\mathcal{D}_{2}\right|$.

$\boxtimes$ tedy označuje množinu všech spojení tohoto typu. V Tutorial D nemá tato operace žádnou analogii. V SQL se používá klíčové slovo\upinlinecode{SQL}{!}{CROSS JOIN}.

\item[Polospojení] Značí se $\ltimes$. Tedy $\mathcal{D}_{1} \ltimes \mathcal{D}_{2} = \pi_{R \cup S} (\mathcal{D}_{1} \Join) \mathcal{D}_{2})$. Výsledkem obsahuje takové n-tice $rs \in \mathcal{D}_{1} \ltimes \mathcal{D}_{2}$, pro které platí, že jsou spojitelné s alespoň jednou n-ticí z $\mathcal{D}_{2}$.
\begin{upcode}{Polospojení (SQL)}{}{SQL}
SELECT DISTINCT table_1.* FROM table_1 NATURAL JOIN table_2
\end{upcode}
\begin{upcode}{Polospojení (Tutorial D)}{}{TutorialD}
relexpr1 MATCHING relexpr2
\end{upcode}
Pokud je $\left\{\!\!\!
\begin{array}{ll}
T = \varnothing & \text{pak } \mathcal{D}_{1} \ltimes \mathcal{D}_{2} = \mathcal{D}_{1} \Join \mathcal{D}_{2} \\
R = \varnothing & \text{pak } \mathcal{D}_{2} \ltimes \mathcal{D}_{1} = \mathcal{D}_{1} \Join \mathcal{D}_{2}
\end{array}\right.$

\item[Restrikce (na rovnost)] $\sigma_{y \approx d} (\mathcal{D}) = \mathcal{D} \Join \left\{ \left\langle y, d \right\rangle \right\} $, kde $\left\langle y, d \right\rangle$ je n-tice nad schématem $R = \left\{ s \right\}$ a $f(y) = d$.
\begin{upcode}{Restrikce na rovnost (Tutorial D)}{}{TutorialD}
relexpr1 WHERE y = d
/* je ekvivalentní s */
relexpr1 JOIN RELATION {TUPLE {y d}}
\end{upcode}

\item[Kompozice binárních relací] Klasické kompozice z pohledu matematických relací se definuje jako:
$$
R_{1} \circ R_{2} = \left\{ \left\langle a, b \right\rangle \; | \; \exists c \text{ tak, že } \left\langle a, b \right\rangle \in R_{1} \text{ a zároveň } \left\langle c, b \right\rangle \in R_{2} \right\}
$$
Naproti tomu kompozice s přihlédnutím k relacím nad určitým relačním schématem se definuje jako:
$$
\mathcal{D}_{1} \circledcirc \mathcal{D}_{2} = \pi_{T \cup R} (\mathcal{D}_{1} \Join \mathcal{D}_{2})
$$
kde relační schémata pro $\mathcal{D}_{1}, \mathcal{D}_{2}$ jsou $\left\{ x, y \right\}$ resp. $\left\{ y, z \right\}$.

\end{description}
\subsubsubsection{Vlastnosti přirozeného spojení}
\begin{upquote}[Komutativita, asociativita, idempotence, neutralita a anihilace při\-/rozeného spojení]
Platí, že:
\begin{align}
\tag{Komutativita}\mathcal{D}_{1} \Join \mathcal{D}_{2} &= \mathcal{D}_{2} \Join \mathcal{D}_{1} \\
\tag{Asociativita}\mathcal{D}_{1} \Join (\mathcal{D}_{2} \Join \mathcal{D}_{3}) &= (\mathcal{D}_{1} \Join \mathcal{D}_{2}) \Join \mathcal{D}_{3} \\
\tag{Idempotence}\mathcal{D} \Join \mathcal{D} &= \mathcal{D} \\
\tag{Neutralita vůči množině obs. $\varnothing$}\mathcal{D} \Join \left\{ \varnothing \right\} &= \mathcal{D} \\
\tag{Neutralita vůči prázdné množině}\mathcal{D} \Join \varnothing &= \varnothing
\end{align}
\end{upquote}
\begin{upquote}[Další vlastnosti spojení]
Platí, že:
\begin{enumerate}
\item $\pi_{R_{j}} (\Join_{i = 1}^{n} \mathcal{D}_{i}) \subseteq \mathcal{D}_{j}, \quad \forall j = 1, \ldots, n$
\item $\pi_{R_{j}} (\Join_{i = 1}^{n} \mathcal{D}_{i}) = \mathcal{D}_{j}, \quad \forall j = 1, \ldots, n$ právě, když lze relace $\mathcal{D}_{1}, \ldots, \mathcal{D}_{n}$ úplně spojit.
\item $\pi_{R_{j}} (\Join_{k = 1}^{n} \pi_{R_{j}} (\Join_{i = 1}^{n} \mathcal{D}_{i})) = \pi_{R_{j}} (\Join_{i = 1}^{n} \mathcal{D}_{i}), \quad j = 1, \ldots, n$
\end{enumerate}
Mějme $\mathcal{D}$ na schématu $R_{1} \cup \cdots \cup R_{i}$. Následně platí:
\begin{enumerate}
\setcounter{enumi}{3}
\item $\mathcal{D} \subseteq \Join_{i = 1}^{n} \pi_{R_{i}} (\mathcal{D})$
\item $\Join_{i = 1}^{n} \pi_{R_{i}} (\Join_{i = 1}^{n} \pi_{R_{i}} (\mathcal{D})) = \Join_{i = 1}^{n} \pi_{R_{i}} (\mathcal{D}) $
\end{enumerate}
\end{upquote}
\begin{uptheorem}[Bezztrátová dekompozice]
Relace $\mathcal{D}$ na schématu $R$ má bezztrátovou dekompozici, pokud
$$
\mathcal{D} = \Join_{i = 1}^{n} \pi_{R_{i}} (\mathcal{D}) \text{ pro } R_{1} \cup \cdots \cup R_{n} = R
$$
\end{uptheorem}
\begin{upnote}[O bezztrátové dekompozici]
Relační schéma $R$ má vždy bezztrátovou dekompozici vzhledem k $R$.
\end{upnote}

\subsubsection{Spojení na rovnost a Théta-spojení}
Spojuje data z tabulek (relací) na základě specifikovaného vztahu, který musí splňovat hodnoty atributů.

Mějme dáno $\mathcal{D}_{1}, \mathcal{D}_{2}$ a obecnou $Theta$-podmínku, jenž je dána obecně nějakou formulí. Pak výsledkem bude podmnožina
$$
\mathcal{D}_{1} \Join_{\Theta} \mathcal{D}_{2} = \sigma_{\Theta} (\mathcal{D}_{1} \boxtimes \mathcal{D}_{2})
$$
kde $\sigma_{\Theta}$ je restrikcí na základě podmínky $\Theta$.

Spojení na rovnost je speciálním případem pro $\Theta$-spojení pro $\Theta$ ve tvaru:
$$
x_{1} = y_{2} \land \cdots \land x_{n} = y_{n} 
$$
\subsubsubsection{Vztah mezi přirozeným spojením a spojením na rovnost}
Spojení na rovnost lze vyjádřit pomocí přirozeného spojení a restrikcí následovně:
$$
\mathcal{D}_{1} \Join_{x_{1} = y_{1} \land \cdots \land x_{n} = y_{n}} \mathcal{D}_{2} = \sigma_{x_{1} = y_{1}} (\sigma_{x_{2} = y_{2}} (\ldots (\sigma_{x_{n} = y_{n}}(\mathcal{D}_{1} \Join \mathcal{D}_{2}))))
$$
Přirozené spojení lze vyjádřit pomocí restrikcí, přejmenování a projekcí. Mějme $\mathcal{D}_{1}, \mathcal{D}_{2}, \ldots, \mathcal{D}_{i} \text{ na } R \cup S \text{ a } \mathcal{D}_{3} \text{ na } S \cup T$. $R, S, T$ jsou po dvou disjunktní pro $S = \left\{ y_{1}, \ldots, y_{n} \right\}$. Bereme $\mathcal{D}_{2}$ a přejmenujeme atributy. Tedy:
$$
\pi_{R \cup S \cup T} (\rho_{y'_{1} \Leftarrow y_{1}, \ldots, y'_{n} \Leftarrow y_{n}} (\mathcal{D}_{2}) \Join_{y_{1} = y'_{1} \land \cdots \land y_{n} = y'_{n}} \mathcal{D}_{1}) = \mathcal{D}_{1} \Join \mathcal{D}_{2}
$$
\begin{upcode}{Vyjádření přirozeného spojení pomocí dalších operací (Tutorial D)}{}{TutorialD}
(relexpr1 JOIN (relexpr2 RENAME (...))) WHERE ...
\end{upcode}
\begin{upcode}{Vyjádření přirozeného spojení pomocí dalších operací (SQL)}{}{SQL}
SELECT * FROM table_1, table_2 WHERE table_1.* = table_2.*;
\end{upcode}