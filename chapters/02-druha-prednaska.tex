\section{Modifikace dat}
Již známe pojmy jako relace nebo relační proměnná. Modifikací dat v databázi se myslí přiřazení nové hodnoty k nějaké relační proměnné. V \upabbrevref{SQL} se k tomuto používají příkazy\upinlinecode{SQL}{!}{INSERT},\upinlinecode{SQL}{!}{UPDATE} nebo\upinlinecode{SQL}{!}{DELETE}. V Tutorial D k obdobným operacím slouží operátor \enquote{:=}, tak jak jej známe například z jazyka Pascal.
\begin{upcode}{Modifikace relace (SQL)}{}{SQL}
UPDATE	zakaznici
SET		plat = 0
WHERE	(plat > 50000);
\end{upcode}
\begin{upcode}{Modifikace relace (Tutorial D)}{}{Tutorial D}
INSERT zakaznici RELATION {TUPLE {id 666, jmeno "Vilík", }};
\end{upcode}
K získávání dat z databáze slouží tzv. \textit{relační dotazování}. Formátování dotazů pak definuje \textit{dotazovací jazyk}. Ten říká, jak se budou dotazy vyhodnocovat. Tyto jazyky jsou obvykle deklarativní. Známe například \upabbrevref{SQL} a Tutorial D.

\textit{Relační algebra} specifikuje množinu operací s relacemmi a dotazy se skládají z postupné aplikace těchto operací. Dotazy se formulují pomocí termů a vyhodnocování dotazů odpovídá vyhodnocování termů v algebře. Vyhodnocování může být přímočaré, avšak u složitějších dotazů rovněž netriviální.

Relační kalkuly (výpočty nad relacemi) existují hned v několika variantách:
\begin{itemize}
\item doménový kalkul
\item n-ticový kalkul
\end{itemize}
Dotaz je formule predikátové logiky, ve které relační symboly označují relační proměnné a vyhodnocování dotazu je ohodnocování formulí v dané predikátové struktuře (ta představuje instanci databáze), ve které jsou relační symboly interpretovány n-árními relacemi.

Doménový relační kalkul má zhruba stejnou sílu jako relační algebra.

\section{Relační algebra}
Relační algebra poskytuje formální podklad pro množinové relační operace. Ty jsou analogií ke standardním operacím na množinách.
\begin{enumerate}
\item \textit{Průnik}, který obsahuje pouze společné n-tice dvou relací. Mějme tedy relace $\mathcal{D}_{1}$ a $\mathcal{D}_{2}$ nad relačním schématem $R \subseteq Y$. Následně průnik definujeme jako:
$$
\mathcal{D}_{1} \cap \mathcal{D}_{2} = \left\{ r \in \prod_{y \in R} dim(y) \; | \; r \in \mathcal{D}_{1} \text{ a zároveň } r \in \mathcal{D}_{2} \right\}
$$
\item \textit{Sjednoceni}, které obsahuje ty n-tice, které se vyskytnou alespoň v jedné ze vstupních relací. Mějme tedy relace $\mathcal{D}_{1}$ a $\mathcal{D}_{2}$ nad relačním schématem $R \subseteq Y$. Následně sjednocení definujeme jako:
$$
\mathcal{D}_{1} \cup \mathcal{D}_{2} = \left\{ r \in \prod_{y \in R} dim(y) \; | \; r \in \mathcal{D}_{1} \text{ nebo } r \in \mathcal{D}_{2} \right\}
$$
\item \textit{Rozdíl}, který obsahuje n-tice, které se nacházejí v první relaci, avšak nenacházejí se relaci druhé. Mějme tedy relace $\mathcal{D}_{1}$ a $\mathcal{D}_{2}$ nad relačním sché-matem $R \subseteq Y$. Následně rozdíl definujeme jako:
$$
\mathcal{D}_{1} \setminus \mathcal{D}_{2} = \left\{ r \in \prod_{y \in R} dim(y) \; | \; r \in \mathcal{D}_{1} \text{ a zároveň } r \notin \mathcal{D}_{2} \right\}
$$
\end{enumerate}
Tyto operace lze provádět také v \upabbrevref{SQL} či v Tutorial D.
\begin{upcode}{Relační operace (Tutorial D)}{}{Tutorial D}
rexpr1 INTERSECT /* nebo UNION či MINUS */ rexpr2
\end{upcode}
\begin{upcode}{Relační operace (SQL)}{}{SQL}
SELECT * FROM table1
	INTERSECT /* nebo UNION či EXCEPT */
SELECT * FROM table2;
\end{upcode}
Množinové operace v \upabbrevref{SQL} používají implicitně volbu\upinlinecode{SQL}{!}{DISTINCT}, takže duplicitní n-tice jsou ignorovány. Naproti tomu u operací typu\upinlinecode{SQL}{!}{SELECT} se jako výchozí používá\upinlinecode{SQl}{!}{ALL}.

\subsection{Efektivita množinových operací}
Efektivitá závisí na implementaci fyzciké vrstvý databázového systému. Obecně platí, že pokud lze na množině potřebných n-tic zavést lineární uspořádání, tak lze množinové operace provádět pomocí slévání.

Je třeba aby dané uspořádání bylo navíc totální. Tedy každé dvě n-tice musejí být porovnatelné. Relace uspořádání musí být tranzitivní, symetrická a reflexivní. Z požadavku totality musí být uspořádání také úplné.

Pokud máme na každé doméně zavedeno toto uspořádání $\leq y$ a zavedeme uspořádání také pro dané relační schéma $R$, tedy $\leq R$, pak nám všechna uspořá-dání
$$
\leq_{y}(y \in R) \text{ a } \leq_{R}
$$
indukují uspořádání na n-ticích $r_{1} < r_{2}$ právě tehdy, když existuje atribut $y \in R$ tak, že
$$
r_{1}(z) = r_{2}(z) \text{ pro každé } z <_{R} y \text{ a navíc } r_{1}(y) <_{y} r_{2}(y)
$$
V \upabbrevref{SQL} se efektivita zajišťuje použitím tzv. indexu. Index dokáže provést ono totální uspořádání.
\begin{upcode}{Tvorba indexu (SQL)}{}{SQL}
CREATE INDEX muj_index ON moje_tabulka (sloupec_1, sloupec_2);
\end{upcode}

\subsection{Další pojmy}
Je třeba vysvětlit několik dalších pojmů:
\begin{description}
\item[Nadklíč (superkey)] Mějme relaci $\mathcal{D}$ nad relačním schématem $R$, která obsahuje několik n-tic. Jelikož je $\mathcal{D}$ množina, tak nemůže obsahovat dvě stejné n-tice. Tedy dvě n-tice, které jsou shodné na všech atributech z $R$. Navíc mohou přirozeně existovat další podmnožiny $R$ (i $R \subseteq R$), pro které platí, že žádné dvě n-tice by se na těchto podmnožinách neměli shodovat. Každou takovou podmnožinu $R$ nazýváme \textit{superklíč}. Každá relace má alespoň jeden superklíč -- ten, který je tvořen všemi atributy.
\item[Klíč (key)] Mějme libovolný nadklíč a relaci $\mathcal{D}$. Takový nadklíč může obsahovat \enquote{nadbytečné} prvky resp. atributy. Jako \textit{klíč} $K$ označíme superklíč takový, že odstraněním jakéhokoliv dalšího atributu z klíče $K$ by vedlo k tomu, že $K$ již nebude nadklíč. Tedy:
\begin{enumerate}
\item Pro každé různé n-tice musí platit, že nemohou mít shodné hodnoty na (všech) atributech z klíče.
\item Klíč je minimálním nadklíčem.
\end{enumerate}
\end{description}

\subsection{Relační operace}
\subsubsection{Projekce}
Mějme relací $R$ na schématu $T$. Projekce $\mathcal{D}$ na $R \subseteq T$ je relace $\pi_{R}(\mathcal{D})$, definovaná jako:
$$
\pi_{R}(\mathcal{D}) = \left\{ r \in \prod_{y \in R} dim(y) \; | \; \text{existuje } s \in \prod_{y \in \pi_{R}} \text{tak, že } rs \in \mathcal{D} \right\}
$$
kde $rs$ znázorňuje zřetězení n-tic, kde část $r \in R$ a $s \in T \setminus R$.

V Tutorial D lze n-tice řetězit prostřednictvým klíčového slova\upinlinecode{TutorialD}{!}{COMPOSE}.
\begin{upcode}{Projekce (SQL)}{}{SQL}
/* část id, jmeno, plat je projekcí */
SELECT DISTINCT id, jmeno, plat FROM table_1;
\end{upcode}
\begin{upcode}{Projekce (Tutorial D)}{}{TutorialD}
/* část {id} je projekcí */
TUPLE {id 666, jmeno "Vilík"} {id}
/* vše kromě id */
TUPLE {id 666, jmeno "Vilík"} {ALL BUT id}
\end{upcode}
Mějme na vědomí, že bez\upinlinecode{SQL}{!}{DISTINCT} se ze zřejmých důvodů nejedná o relační operaci.

\subsubsubsection{Aspekty fyzické vrstvy}
Projekce v SQL přes\upinlinecode{SQL}{!}{DISTINCT} je rychlá operace. V případě užití\upinlinecode{SQL}{!}{DISTINCT} se už o rychlou operaci jednat nemusí. To ale neplatí v případě, že $R \subseteq T$ obsahuje klíč. Projekcí umí zrychlit:
\begin{enumerate}
\item Hashování, kde se potencíální n-tice, které se objeví ve výsledku, hashují.
\item Použití totálního uspořádání n-tic.
\begin{upquote}[Užití totálního uspořádání pro zrychlení projekce]
Pokud máme $S \subseteq R \subseteq T$, pak
\begin{align*}
\pi_{S} (\pi_{R}(\mathcal{D})) &= \pi_{S}(\mathcal{D}) \\
\pi_{T} (\mathcal{D}) &= \mathcal{D} \\
\pi_{R} (\mathcal{D}) &=\left\{\!\!\!
\begin{array}{ll}
\varnothing & \text{pokud } \mathcal{D} = \varnothing \\
\left\{ \varnothing \right\} & \text{pokud } \mathcal{D} \neq \varnothing
\end{array}\right.
\end{align*}
\end{upquote}
\end{enumerate}

\subsubsection{Restrikce (selekce)}
Mějme danou relaci $\mathcal{D}$ a formulí popisující podmínku $\varphi : t_{1} \approx t_{2}$. $t_{1}$ je atribut a $t_{2}$ je hodnota z domény atributu. Následně restrikce:
$$
\sigma_{\varphi}(\mathcal{D}) = \left\{ t \in \mathcal{D} \; | \; t \text{ splňuje } \varphi \right\}
$$
\begin{upcode}{Restrikce (SQL)}{}{SQL}
/* část id = 666 je restrikcí */
SELECT * FROM table_1 WHERE id = 666;
\end{upcode}
\begin{upcode}{Restrikce (Tutorial D)}{}{TutorialD}
/* část id = 666 je restrikcí */
relexpr WHERE id = 666;
\end{upcode}
\begin{upquote}[O komutativitě restrikcí]
Platí, že:
$$
\sigma_{\varphi}(\sigma_{\psi}(\mathcal{D})) = \sigma_{\psi}(\sigma_{\varphi}(\mathcal{D})) = \sigma_{\varphi \times \psi}(\mathcal{D})
$$
\end{upquote}

\subsubsubsection{Záměna pořadí projekce a restrikce}
Prohlašme, že $\pi_{R} (\sigma_{\varphi} (\mathcal{D})) \backsim \sigma_{\varphi} (\pi_{R} (\mathcal{D}))$. Pokud platí, že $\varphi$ závisí na některém atributu z $T \setminus R$, pak pravá strana nedává smysl.

Předchozí poznatek má použití, a sice duální operaci k projekci, tedy přidání atributů.
\begin{upcode}{Přidání atributů (SQL)}{}{SQL}
SELECT table_1.*, expr AS additional_column FROM table_1;
\end{upcode}
\begin{upcode}{Přidání atributů (Tutorial D)}{}{TutorialD}
EXTEND relexpr ADD (expr AS additional_column)
\end{upcode}
\begin{upcode}{Kombinace restrikce a duality projekce (SQL)}{}{SQL}
SELECT	money * 2 AS doubled_money, money, id
FROM 	employees
WHERE	money * 2 > 5000;
\end{upcode}
\begin{upcode}{Kombinace restrikce a duality projekce (Tutorial D)}{}{TutorialD}
((EXTEND relexpr ADD (money * 2 AS doubled_money))
WHERE doubled_money > 5000) {doubled_money, money, id}
\end{upcode}